%%%%%%%%%%%%%%%%%%%%%%%%%%%%%%%%%%%%%%%%%%%%%%%%%%
\documentclass{beamer}

\mode<presentation> {

\usepackage{pgfgantt}
\usetheme{Madrid}
}

\usepackage{graphicx} 
\usepackage{booktabs} 
\RequirePackage{ifthen} % package required

\newboolean{sectiontoc}
\setboolean{sectiontoc}{true} % default to true

\title[Statistical Student Modeling]{Statistical Student Modeling}
\logo{\includegraphics[width=.8cm,keepaspectratio]{logo.png}}
%\author[Pooja Agarwal]
\author[]{Rahul Yedida
\newline
Ankush Kumar
\newline
Srihari Joshi
\newline	
Vima Rai} 
\institute[Batch 48]{Ms. Kundhavai K R
\newline
Batch No. - 48}
\date{\today} 

\newcommand{\sectiontoc}{
	\begin{frame}
		\frametitle{Overview}
		\tableofcontents[currentsection]
	\end{frame}
}

\newcommand{\subsectiontoc}{
	\begin{frame}
		\frametitle{Overview}
		\tableofcontents[currentsection,currentsubsection]
	\end{frame}
}

\begin{document}

\begin{frame}
\titlepage 
\end{frame}

\section{Introduction}
\sectiontoc

\begin{frame}
\frametitle{Problem Statement / Definition} 
\begin{itemize}
		\item<1-> \textbf{Domain:} Educational data mining, statistical learning
		\item<2-> \textbf{What:} An Intelligent Tutoring System (ITS)
		\item<3-> \textbf{How:} Several algorithms proposed in literature, based on BKT
		\item<4-> \textbf{Data:} 2009-10 Skill-builder ASSISTments data
		\item<5-> \textbf{Metrics:} RMSE, MAE
	\end{itemize}
\end{frame}

\begin{frame}{Intelligent Tutoring Systems}
	\begin{itemize}
		\item<1-> Adaptive teaching systems for elucidating concepts
		\item<2-> Primarily based on Hidden Markov Models (HMMs)
		\item<3-> Generated interest after Corbett \& Anderson, 1994.
	\end{itemize}
\end{frame}

\begin{frame}{Motivation}
	\begin{itemize}
		\item<1-> Model students learning state
		\item<2-> Use non-traditional cues, e.g. affect
		\item<3-> Can modeling help improve education?
	\end{itemize}
\end{frame}

\begin{frame}{So what are we doing?}
	\begin{itemize}
		\item<1-> Implement a web-based ITS solution
		\item<2-> Individual models for each user
		\item<3-> Idea: start with simple models (single concept, basic BKT), go increasingly complex, hopefully implement KAT.
	\end{itemize}
\end{frame}

\section{Literature Survey}
\subsection{Algorithm}
\subsectiontoc

\begin{frame}{Bayesian Knowledge Tracing (BKT)}
	\begin{itemize}
		\item<1-> Proposed by Corbett \& Anderson, 1994.
		\item<2-> Fundamentally, a two-state HMM--\textit{learned} and \textit{unlearned}.
		\item<3-> Viterbi algorithm can be used to solve for the hidden state sequence.
	\end{itemize}
\end{frame}

\subsection{Extensions}
\subsectiontoc

\begin{frame}{BKT Extensions}
	\begin{itemize}
		\item Pardos and Heffernan, 2011. Incorporated problem difficulty.
		\item Yudelson et al., 2013. Incorporated student learning speed.
		\item Schultz and Arroyo, 2014. Combined BKT with HMM-IRT, called Knowledge and Affect Tracing (KAT) model.
		\item Lin and Chi, 2016. Added student response time directly into the model, creating the Intervention-BKT (I-BKT).
		\item Spaulding, Gordon, Brezeal, 2016. Used commercial affect-analysis tool called Affdex.
	\end{itemize}
\end{frame}

\subsection{Alternatives}
\subsectiontoc

\begin{frame}{Why not Deep Neural Networks?}
	\begin{itemize}
		\item<1-> RNNs, LSTMs successfully applied (Piech et al., 2015; Lin and Chi, 2017)
		\item<2-> Difficult to interpret!
		\item<3-> With HMMs, can identify "most likely" hidden state sequence, and can also find HMM parameters (EM algorithm)
	\end{itemize}
\end{frame}

\section{Requirements}
\subsection{Hardware}
\subsectiontoc

\begin{frame}{User}
	\begin{itemize}
		% These are obviously to lighten up the mood.
		% It's a web app, so there are no hardware requirements
		% for the end user.
		\item Working router
		\item Computer
	\end{itemize}
\end{frame}

\begin{frame}{Server}
	\begin{itemize}
		\item 2 GB RAM
		\item Optional: GPU, if using affect-aware models
	\end{itemize}
\end{frame}

\subsection{Software}
\subsectiontoc

\begin{frame}{User}
	Recent web browser
\end{frame}

\begin{frame}{Server}
	\begin{itemize}
		\item Python, Flask
		\item Node.js, npm
		\item pycodestyle
		\item GNU/Linux
	\end{itemize}
\end{frame}

\section{Design}
\sectiontoc

\begin{frame}{High level design}
\includegraphics[scale=0.5]{high_level.png}
\end{frame}

\begin{frame}{Sequence diagram}
\includegraphics[scale=0.4]{Login.png}
\end{frame}

\begin{frame}{Sequence diagram}
\includegraphics[scale=0.3]{Working.png}
\end{frame}

\section{Timeline}
\subsection{Back End}
\subsectiontoc

\begin{frame}
\frametitle{Timeline from Sep 12 (first commit) to Oct 24 (F3)}
\definecolor{barblue}{RGB}{153,204,254}
\definecolor{groupblue}{RGB}{51,102,254}
\definecolor{linkred}{RGB}{165,0,33}
\renewcommand\sfdefault{phv}
\renewcommand\mddefault{mc}
\renewcommand\bfdefault{bc}
\setganttlinklabel{s-s}{START-TO-START}
\setganttlinklabel{f-s}{FINISH-TO-START}
\setganttlinklabel{f-f}{FINISH-TO-FINISH}
\sffamily

\begin{ganttchart}[
    canvas/.append style={fill=none, draw=black!5, line width=.75pt},
    hgrid style/.style={draw=black!5, line width=.75pt},
    vgrid={*1{draw=black!5, line width=.75pt}},
    today=6,
    today rule/.style={
      draw=black!64,
      dash pattern=on 3.5pt off 4.5pt,
      line width=1.5pt
    },
    today label font=\small\bfseries,
    title/.style={draw=none, fill=none},
    title label font=\bfseries\footnotesize,
    title label node/.append style={below=7pt},
    include title in canvas=false,
    bar label font=\mdseries\small\color{black!70},
    bar label node/.append style={left=1cm},
    bar/.append style={draw=none, fill=black!63},
    bar incomplete/.append style={fill=barblue},
    bar progress label font=\mdseries\footnotesize\color{black!70},
    group incomplete/.append style={fill=groupblue},
    group left shift=0,
    group right shift=0,
    group height=.5,
    group peaks tip position=0,
    group label node/.append style={left=.6cm},
    group progress label font=\bfseries\small,
    link/.style={-latex, line width=1.5pt, linkred},
    link label font=\scriptsize\bfseries,
    link label node/.append style={below left=-2pt and 0pt}
  ]{1}{13}
  \gantttitle[
    title label node/.append style={below left=7pt and -3pt}
  ]{FORTNIGHT:\quad1}{1}
  \gantttitlelist{2,...,10}{1} \\
  
  % PART 1: SETUP
  \ganttgroup[progress=100]{Setup}{1}{2} \\
  
  \ganttbar[
    progress=100
  ]{\textbf{1.1} Download data}{1}{1} \\
  
  \ganttbar[
    progress=100
  ]{\textbf{1.2} Hello World}{1}{2} \\
  
  % PART 2: STANDARD BKT MODEL
  \ganttgroup[progress=100]{Standard BKT Model}{2}{3} \\
  
  \ganttbar[
    progress=100
  ]{\textbf{2.1} Implement Python class}{2}{2} \\
  
  \ganttbar[
    progress=100
  ]{\textbf{2.2} EDA}{3}{3} \\
  
   \\[grid]
  
\end{ganttchart}
\end{frame}


\begin{frame}
\frametitle{Timeline from Oct 24, 2018 (F4) to April 10, 2019 (F15)}
\definecolor{barblue}{RGB}{153,204,254}
\definecolor{groupblue}{RGB}{51,102,254}
\definecolor{linkred}{RGB}{165,0,33}
\renewcommand\sfdefault{phv}
\renewcommand\mddefault{mc}
\renewcommand\bfdefault{bc}
\setganttlinklabel{s-s}{START-TO-START}
\setganttlinklabel{f-s}{FINISH-TO-START}
\setganttlinklabel{f-f}{FINISH-TO-FINISH}
\sffamily

\begin{ganttchart}[
    canvas/.append style={fill=none, draw=black!5, line width=.75pt},
    hgrid style/.style={draw=black!5, line width=.75pt},
    vgrid={*1{draw=black!5, line width=.75pt}},
    today=2,
    today rule/.style={
      draw=black!64,
      dash pattern=on 3.5pt off 4.5pt,
      line width=1.5pt
    },
    today label font=\small\bfseries,
    title/.style={draw=none, fill=none},
    title label font=\bfseries\footnotesize,
    title label node/.append style={below=7pt},
    include title in canvas=false,
    bar label font=\mdseries\small\color{black!70},
    bar label node/.append style={left=1cm},
    bar/.append style={draw=none, fill=black!63},
    bar incomplete/.append style={fill=barblue},
    bar progress label font=\mdseries\footnotesize\color{black!70},
    group incomplete/.append style={fill=groupblue},
    group left shift=0,
    group right shift=0,
    group height=.5,
    group peaks tip position=0,
    group label node/.append style={left=.6cm},
    group progress label font=\bfseries\small,
    link/.style={-latex, line width=1.5pt, linkred},
    link label font=\scriptsize\bfseries,
    link label node/.append style={below left=-2pt and 0pt}
  ]{1}{13}
  \gantttitle[
    title label node/.append style={below left=7pt and -3pt}
  ]{FORTNIGHT:\quad1}{0}
  \gantttitlelist{4,...,15}{1} \\
  
  % PART 3: I-BKT
  \ganttgroup[progress=0]{I-BKT}{1}{1} \\
  
  % PART 4: KAT MODEL
  \ganttgroup[progress=0]{KAT Model}{7}{9} \\
  
  \ganttbar[
    progress=0
  ]{\textbf{4.1} Implement affect analysis}{7}{8} \\
  
  \ganttbar[
    progress=0
  ]{\textbf{4.2} Implement model}{9}{9} \\
  
   \\[grid]
  
\end{ganttchart}
\end{frame}

\subsection{Front end}
\subsectiontoc

\begin{frame}
\frametitle{Timeline from Sep 12 (first commit) to Oct 24 (F3)}
\definecolor{barblue}{RGB}{153,204,254}
\definecolor{groupblue}{RGB}{51,102,254}
\definecolor{linkred}{RGB}{165,0,33}
\renewcommand\sfdefault{phv}
\renewcommand\mddefault{mc}
\renewcommand\bfdefault{bc}
\setganttlinklabel{s-s}{START-TO-START}
\setganttlinklabel{f-s}{FINISH-TO-START}
\setganttlinklabel{f-f}{FINISH-TO-FINISH}
\sffamily

\begin{ganttchart}[
	canvas/.append style={fill=none, draw=black!5, line width=.75pt},
	hgrid style/.style={draw=black!5, line width=.75pt},
	vgrid={*1{draw=black!5, line width=.75pt}},
	today=6,
	today rule/.style={
		draw=black!64,
		dash pattern=on 3.5pt off 4.5pt,
		line width=1.5pt
	},
	today label font=\small\bfseries,
	title/.style={draw=none, fill=none},
	title label font=\bfseries\footnotesize,
	title label node/.append style={below=7pt},
	include title in canvas=false,
	bar label font=\mdseries\small\color{black!70},
	bar label node/.append style={left=1cm},
	bar/.append style={draw=none, fill=black!63},
	bar incomplete/.append style={fill=barblue},
	bar progress label font=\mdseries\footnotesize\color{black!70},
	group incomplete/.append style={fill=groupblue},
	group left shift=0,
	group right shift=0,
	group height=.5,
	group peaks tip position=0,
	group label node/.append style={left=.6cm},
	group progress label font=\bfseries\small,
	link/.style={-latex, line width=1.5pt, linkred},
	link label font=\scriptsize\bfseries,
	link label node/.append style={below left=-2pt and 0pt}
	]{1}{13}
	\gantttitle[
	title label node/.append style={below left=7pt and -3pt}
	]{FORTNIGHT:\quad1}{1}
	\gantttitlelist{2,...,10}{1} \\
	
	% PART 1: Setup
	\ganttgroup[progress=100]{Setup}{1}{2} \\
	
	\ganttbar[
	progress=10
	]{\textbf{1.1} Sign-up}{3}{3} \\
	
	\ganttbar[
	progress=50
	]{\textbf{1.2} Log-in}{2}{3} \\
	
	% PART 2: Landing page
	\ganttgroup[progress=0]{Landing Page}{3}{3} \\
	
	\ganttbar[
	progress=0
	]{\textbf{2.1} Landing page}{3}{3} \\
	
	\\[grid]
\end{ganttchart}
\end{frame}


\begin{frame}
\frametitle{Timeline from Oct 24, 2018 (F4) to April 10, 2019 (F15)}
\definecolor{barblue}{RGB}{153,204,254}
\definecolor{groupblue}{RGB}{51,102,254}
\definecolor{linkred}{RGB}{165,0,33}
\renewcommand\sfdefault{phv}
\renewcommand\mddefault{mc}
\renewcommand\bfdefault{bc}
\setganttlinklabel{s-s}{START-TO-START}
\setganttlinklabel{f-s}{FINISH-TO-START}
\setganttlinklabel{f-f}{FINISH-TO-FINISH}
\sffamily

\begin{ganttchart}[
canvas/.append style={fill=none, draw=black!5, line width=.75pt},
hgrid style/.style={draw=black!5, line width=.75pt},
vgrid={*1{draw=black!5, line width=.75pt}},
today=2,
today rule/.style={
	draw=black!64,
	dash pattern=on 3.5pt off 4.5pt,
	line width=1.5pt
},
today label font=\small\bfseries,
title/.style={draw=none, fill=none},
title label font=\bfseries\footnotesize,
title label node/.append style={below=7pt},
include title in canvas=false,
bar label font=\mdseries\small\color{black!70},
bar label node/.append style={left=1cm},
bar/.append style={draw=none, fill=black!63},
bar incomplete/.append style={fill=barblue},
bar progress label font=\mdseries\footnotesize\color{black!70},
group incomplete/.append style={fill=groupblue},
group left shift=0,
group right shift=0,
group height=.5,
group peaks tip position=0,
group label node/.append style={left=.6cm},
group progress label font=\bfseries\small,
link/.style={-latex, line width=1.5pt, linkred},
link label font=\scriptsize\bfseries,
link label node/.append style={below left=-2pt and 0pt}
]{1}{13}
\gantttitle[
title label node/.append style={below left=7pt and -3pt}
]{FORTNIGHT:\quad1}{0}
\gantttitlelist{4,...,15}{1} \\

% PART 3: Landing page
\ganttgroup[
progress=0
]{Landing page}{1}{1} \\

% PART 4: ITS
\ganttgroup[
progress=0
]{ITS}{4}{13} \\

\\[grid]

\end{ganttchart}
\end{frame}

\begin{frame}
\frametitle{References}
\begin{footnotesize}
	\begin{thebibliography}{99}
		\bibitem{p1} A. T. Corbett and J. R. Anderson (1994)
		\newblock Knowledge tracing: Modeling the acquisition of procedural knowledge
		\newblock \emph{User modeling and user-adapted interaction,} 4(4), 253 -- 278.
		
		\bibitem{p1} R. S. d Baker, A. T. Corbett, and V. Aleven (2008)
		\newblock More accurate student modeling through contextual estimation of slip and guess probabilities in bayesian knowledge tracing
		\newblock \emph{Intelligent Tutoring Systems,} 253 -- 278, Springer.
		
		\bibitem{p1} C. Lin and M. Chi (2016)
		\newblock Intervention-bkt: incorporating instructional interventions into bayesian knowledge tracing
		\newblock \emph{International Conference on Intelligent Tutoring Systems,} 208 -- 218, Springer.
		
		\bibitem{p1} S. Chiappa and S. Bengio (2003)
		\newblock Hmm and iohmm modeling of eeg rhythms
for asynchronous bci systems
		\newblock \emph{Tech. rep.,} IDIAP.
	\end{thebibliography}
\end{footnotesize}
\end{frame}

\begin{frame}
\frametitle{References}
\begin{footnotesize}
	\begin{thebibliography}{99}
		\bibitem{p1} S. Spaulding, G. Gordon, and C. Breazeal (2016)
		\newblock Affect-aware student models for robot tutors
		\newblock \emph{Proceedings of the 2016 International Conference on Autonomous Agents \& Multiagent Systems,} 864 -- 872, International
Foundation for Autonomous Agents and Multiagent Systems.
		
		\bibitem{p1} S. Schultz and I. Arroyo (2014)
		\newblock Tracing knowledge and engagement in parallel in an intelligent tutoring system
		\newblock \emph{Educational Data Mining}.
		
		\bibitem{p1} Z. A. Pardos and N. T. Heffernan (2011)
		\newblock Kt-idem: introducing item difficulty to the knowledge tracing model
		\newblock \emph{International Conference on User Modeling, Adaptation, and Personalization,} 243 -- 254, Springer.
		
		\bibitem{p1} M. V. Yudelson, K. R. Koedinger, and G. J. Gordon (2013)
		\newblock Individualized bayesian knowledge tracing models
		\newblock \emph{International Conference on Artificial Intelligence in Education,} 171 -- 180, Springer.
	\end{thebibliography}
\end{footnotesize}
\end{frame}

\begin{frame}
\frametitle{References}
\begin{footnotesize}
	\begin{thebibliography}{99}
		\bibitem{p1} C. Piech, J. Bassen, J. Huang, S. Ganguli, M. Sahami, L. J. Guibas, and J. Sohl-Dickstein, (2015)
		\newblock Deep knowledge tracing
		\newblock \emph{Advances in Neural Information Processing Systems,} 505 -- 513.
		
		\bibitem{p1} C. Lin and M. Chi (2017)
		\newblock A comparisons of bkt, rnn and lstm for learning gain prediction
		\newblock \emph{International Conference on Artificial Intelligence in Education,} 536 -- 539, Springer.
		
		\bibitem{p1} Z. C. Lipton, D. C. Kale, C. Elkan, and R. Wetzel (2015)
		\newblock Learning to diagnose with lstm recurrent neural networks
		\newblock \emph{arXiv preprint,} arXiv:1511.03677.
		
		\bibitem{p1} J. Johns and B. Woolf (2006)
		\newblock A dynamic mixture model to detect student motivation and proficiency
		\newblock \emph{Proceedings of the National Conference on Artificial Intelligence,} 21(1), 163, AAAI Press; MIT Press.
	\end{thebibliography}
\end{footnotesize}
\end{frame}

%------------------------------------------------

\begin{frame}
\Huge{\centerline{The End}}
\end{frame}

%----------------------------------------------------------------------------------------

\end{document}
